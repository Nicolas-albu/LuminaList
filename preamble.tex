\usepackage[T1]{fontenc}
\usepackage{lmodern}
% \usepackage{graphicx}
\usepackage{enumitem}
\usepackage{tcolorbox}
\tcbuselibrary{breakable}
\usepackage{xcolor}
\usepackage{geometry}

% Pacote apenas para demonstração, pode remove-lo.
\usepackage{lipsum}

% para disciplina de "Algoritmos e Estrutura de Dados"
% \usepackage{algpseudocode}
% \usepackage{algorithm}

% para cálculo e símbolos matemáticos
\usepackage{amsfonts}
\usepackage{amsmath}
\usepackage{amssymb}
\usepackage{mathtools}
\usepackage{stmaryrd}
\usepackage{calc}


% CONFIGURAÇÕES GERAIS
% Ajusta a margem
\geometry{
    right=1.5cm,
    left=1.5cm,
    bottom=2cm,
    top=2cm,
}

% Configura a quebra do tcolorbox
\tcbset{breakable, pad at break=2mm}


% AMBIENTES
% Defina um novo ambiente "exercise"
\newenvironment{exercise}[2][]{
    \begin{tcolorbox}[
        colback=blue!20!white,
        colframe=blue!40!black,
        ]
        \ifx\relax#1\relax
            \textbf{\textcolor{blue!40!black}{Exercício #2}}
        \else
            \textbf{\textcolor{blue!40!black}{Exercício #2: #1}}
        \fi
}{
    \end{tcolorbox}
}

% Definindo o ambiente "solution" padrão
\newenvironment{solution}[1][]{
    \begin{tcolorbox}[
        colback=green!15!white,
        colframe=green!40!black,
        ]
        \ifx\relax#1\relax
            % Se não houver um título personalizado
        \else
            \textbf{Solução #1}
        \fi
}{
    \end{tcolorbox}
}

% Definindo o ambiente "solutionitem" para itens
\newenvironment{solutionitem}[2][]{
    \begin{tcolorbox}[
        colback=green!15!white,
        colframe=green!40!black,
        ]
        \begin{enumerate}
            \item[(#1)] #2
        \end{enumerate}

        \vspace{1.5ex}\textbf{Solução}
}{
    \end{tcolorbox}
}
