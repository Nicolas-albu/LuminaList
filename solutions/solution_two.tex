\begin{solutionitem}[a]{Mostre que o operador linear $F$ do $\mathbb{R}^3$ dado por $F(x, y, z) = (x + y - z, x - z, y + z)$ é um automorfismo.}
    Para provar que $F$ é automorfismo em $\mathbb{R}^3$, então temos que provar a sua \textit{bijetividade} que é consequência da \textit{injetividade} e \textit{sobrejetividade} de $F$.
        \begin{enumerate}
            \item[(i)] Para mostrar que $F$ é \textit{injetora}, basta determinar o $\ker(F)$. Um elemento $(x,y) \in \mathbb{R}^3$, se,
            \begin{align*}
                F(x, y, z) &= (x + y - z, x - z, y + z) = (0, 0, 0) \\
                &\Longleftrightarrow 
            \left\{
                \begin{matrix}
                    x & + & y & - & z & = & 0 \\
                    x & - & y &   &   & = & 0 \\
                      &   & y & + & z & = & 0 \\
                \end{matrix}
            \right.
            \hspace{0.5cm}\rightsquigarrow\hspace{0.5cm}
            \left\{
                \begin{matrix}
                    x & = & 0 \\
                    y & = & 0 \\
                    z & = & 0 \\
                \end{matrix}
            \right.
            \end{align*}

            Assim, concluimos que $\ker(F) = \{(0, 0, 0)\}$. Consequentemente, \textbf{descobrimos que $F$ é injetora}.

        \item[(ii)] Utilizando o teorema do núcleo e da imagem, temos,
        \[ 
        \dim(\mathbb{R}^3) = \dim(\ker(F)) + \dim(\Im(F)) \Longrightarrow \dim(\mathbb{R}^3) = \dim(\Im(F))
        \]

        Desse modo, \textbf{observamos que $F$ é sobrejetivo}.
        \end{enumerate}
        Logo, $F$ é bijetora, e consequentemente é um \textit{automorfismo}. $\hfill{\blacksquare}$
\end{solutionitem}

\begin{solutionitem}[b]{\lipsum[5-5]}
    \lipsum[5-6]
\end{solutionitem}